\documentclass[a4paper,oneside, 12pt]{report}
\usepackage[latin1]{inputenc}
\usepackage{multicol}
\usepackage{graphicx} 
\usepackage{float}
\usepackage{fancyhdr}
\pagestyle{fancy}
\usepackage{array}
\fancyhf{}
\lhead[\thepage]{\rightmark}
\pagestyle{headings}
\usepackage[top=3cm, bottom=3cm, left=2.5cm, right=2.5cm]{geometry}
\renewcommand{\thesection}{\arabic{section}}
\usepackage{enumerate}
\usepackage{mathtools}
\usepackage{fixltx2e}
\usepackage{hyperref}

\usepackage{booktabs} % nicer tables, use \toprule \midrule \bottomrule instead of \hline
\usepackage{topcapt} % same, use \topcaption{} as caption above table inseat of \caption{}
\usepackage{multirow}
\usepackage{tikz} % for latex-made figures
\usepackage{pgfplots}
\usetikzlibrary{shapes,arrows}
\usepackage{subfigure}
\usepackage{url}


\begin{document}

\begin{titlepage}
\vspace{10cm}
\begin{center}
\large
R programming\\
Notes
\end{center}

\vspace*{3cm}


\end{titlepage}

%-------------------------------------------------------------------------------------
\setcounter{tocdepth}{3}
\tableofcontents
\setcounter{page}{2}
\pagebreak

\newpage
\section{Lecture 1}
\subsection{Reading and writing data}
Function name: read.table() \\
Input arguments: file name, header = logical index if the file has a header line, sep = a string indicating how the columns are separated. \\
\\
Function read.csv() is identical to read.table except that the default separator is comma while for read.table the separator is space.\\
\\
Function name: readLines(); it is used to read text lines \\
\\
When the dataset is loaded in the R, it is stored in RAM so it is important to have rough estimate of the data size. Another important argument as colClasses can be used to specify of different data class and R does not have to deal with it automatically that makes the program slow when dealing with large datasets. \\
{\bf Textual formatting:} Dumping and Dputing provides the editable textual formats. dput constructs the R code that can be used to get the R object. Similarly multiple objects can be departed using the dump function and read back in using source. \\
{\bf Interfaces:} Data are read in using connection interfaces. Connection can made to files or to other sources like webpages. for example file is used to open connections to a file, url to open a connection to we page.\\
\\
Function name: file()
Input arguments: ``r'' - read only, ``w'' - write only etc.
\section{Lecture 2}
\subsection{Control structure}
{\bf IF loop:}\\
General loop structure:- \\
if(condition)\{\\
\indent expression\\
\}\\
else if(condition)\{\\
\indent expression\\
\}\\
else \{\\
\}\\
\\
{\bf FOR loop:} For loop can be nested, so a loop can be inside another loop.\\
General loop structure:- \\
for(i in 1:10)\{\\
\indent expressions\\
\}\\
\\
{\bf While loop:} It begins by testing a condition. If it is true, the code is executed. While loops can potentially result in infinite loop so one need to be careful when executing while loop.\\
General loop structure:- \\
while(condition)\{\\
\indent Expressions\\
\indent condition control structure\\
\}\\
\\
{\bf Repeat loop:} Repeat initiates an infinite loop. The only way to exit a repeat loop is to call break. It is better to use for loop than using the repeat loop.\\
General loop structure:- \\
repeat \{\\
\indent Expression\\
\indent condition of convergence and break \\
\}\\
\\
{\bf Next:} It is used to skip an iteration of a loop.\\
General loop structure:- \\
for (i in 1:100)\{\\
\indent if(i <= 20)\{\\
\indent \indent next\\
\indent \}\\
\indent Expression\\
\}\\
\subsection{Functions}
Function are first class object which means that they can passed as arguments to other functions or it can be nested.\\
General loop structure:- \\
f <- function(arguments)\{\\
expressions\\
\}\\
Function arguments can have the default arguments. If the arguments are named, the order of the arguments can be reversed however having one named argument and another unnamed argument can lead to confusion.
To see the arguments of a function type-->\\
args(function name)\\
\\
... represents a variable number of arguments that are usually passed on to other functions. \\
args(paste) -- Paste function allows you to paste different variables. It starts with ... as different arguments can be included in the paste function. However after ... you can use partial matching of the variables. By default sep paste space between the arguments.\\

\section{Lecture 3}
{\bf Loop functions:} Loop functions have apply as the keyword in them. These looping functions make the implementation of loops easier. \\
\\
{\bf lapply:} It loops over a list and evaluate a function on each element. lapply always returns a list, regardless of the class of the input.\\
Function name: lapply\\
Arguments: (1) a list x (2) a function FUN and (3) other arguments via its ... arguments.\\
Example code: \\
x $<-$ 1:4 \\
lapply(x, runic, min= 0, max = 10)\\
lapply and friends make heavy use of anonymous functions. Anonymous functions are those which do not have a name. For example a function for extracting the first column of each matrix.\\
\indent \indent lapply(x, function(elt) elt[,1])
\\
\\
{\bf sapply:} A variant of lapply that simplifies the results of the lapply. If the result of lapply is a list where every element is length 1, then a vector is returned for spply. If the result is a list where every element is a vector of the same length, a matrix is returned. \\
Example: sapply(x, mean)
\\
\\
{\bf Apply:} Apply function is applied on array, to apply the function across the entire array.\\
Example: apply(x, 2, mean) where 2 is used to get the column wise mean.\\
\indent apply(x,1, quantile, probs = c(0.25, 0.75))\\

To calculate row mean or row sums we have functions like, rowSums, rowMeans, colSums etc. They are faster than using apply function. 
\\
\\
{\bf tappply} It is used to apply a function over subsets of a vector. \\
Function name: str(tapply) \\
usage: function(X, INDEX, fun = NULL, ... , simplify = TURE)\\
Example: x $<-$ c(rnorm(10), runif(10), rnorm(10,1))\\
\indent f$<-$ gl(3,10)\\
\indent tapply(x, f, mean)\\ Provides the mean for x depending on the levels. If the simplify is not applied we get a list as a solution.
\\
\\
{\bf split:} It takes a vector or other objects and splits it into groups determined by a factor or list of factors. \\
Function name: function(x, f, drop = FALSE, ...)\\
Usage: split(x,f) in above example will give a list, but this does not apply the summary statistics. It splits the x in different levels indicated by levels in f. Once split, tapply and other functions can be used onto it. e.g.\\
\indent lapply(split(x,f), mean) : here tapply will do the same thing. \\
\indent Usage: s $<-$ split(airquality, airquality\$Month)\\
\indent \indent lapply(s, function(x) colMeans(x[,c("Ozone", "Solar.R", "wind")])\\
However it will give a list. Here we use lapply over split along with an anonymous function. \\ 
\\
{\bf mapply}: A multivariate apply function that applies a function in parallel over a set of arguments. It allow one to apply the same functions over different list without using for loop.\\
Function usage: function(FUN, ... , Moreargs = NULL, simplify = TRUE)\\
Example: list ( rep(1,4), rep(2,3). rep(3,2), rep(4,1)) instead we can also do\\
\indent mapply(rep, 1:4, 4:1)\\
Here mapply is used to apply same function over different arguments\\

\subsection{Debugging}
How to get information about errors:
\begin{itemize}
	\item Message: Normal message, diagnostic message
	\item warning: something is wrong but not necessarily fatal, execution of the function continues
	\item error: tatar problem has occurred and the execution of the function stops.
	\item condition: Something unexpected has occur
\end{itemize}
Functions to debug the program:
\begin{itemize}
	\item trackback: prints out the function call stack after an error occurs; does nothing if no error.
	\item debug: flags a function for "debug" mode which allows you to step through execution of a function one line at a time.
	\item browser: suspends the execution of a function whenever it is called and puts the function in debug mode. Can be started anywhere in the function.
	\item trace: allows you to insert debugging code into a function a specific places.
	\item recover: allows you to modify the error behaviour so that you can browse the function call stack.
\end{itemize}
\end{document}
